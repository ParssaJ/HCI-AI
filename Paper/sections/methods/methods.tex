\documentclass[../../submission.tex]{subfiles}
\begin{document}
\section{Methods}
A user study was conducted to address the following research questions:

\begin{itemize}
    \item RQ1:What characterizes typical natural language user queries to SQL systems?
    \item RQ2:What are the factors that influence the mismatch between user intent and displayed search results?
    \item RQ3:How do users prefer to fix SQL queries when they don't match their intentions?
    \item RQ4:What are effective ways to communicate system limitations and capabilities to users?    
    \item RQ5:What mechanisms most influence user satisfaction with search results?
    \item RQ6:What are the potential drawbacks of deploying such a system in a real-world environment?
    \item RQ7:How necessary are filters when an intelligent system is involved?
\end{itemize}
The optimization of SQL queries can be achieved through various methods. 
One such approach entails enhancing the initial query. However, this approach may present a challenge, as it requires that the user comprehend the erroneous interpretation of the query. 
To optimize the request, it is first necessary to develop a more profound comprehension of the underlying system. 
In this regard, the method for enhancing the user experience proposed in \cite{popescuEtalTowardsTheoryOfNaturalLanguage} may be applicable. 
This method enables the user to select from among various options (RQ3).  

One of the objectives of the present study is to ascertain whether users find conventional search results satisfactory or whether they perceive the system 
with artificial intelligence as more advantageous.
We hypothesize that users will evaluate the system employing artificial intelligence more effectively than they would a conventional system.
The reason for this is that users will have more ease of use when filtering content, as opposed to the more difficult experience they face when using a conventional system(RQ5).

The hypothesis that was formulated posits that filters are only required when the system fails to comprehend the user's intention. 
It is hypothesized that both procedures will be utilized, as the prevailing Webshop practice involves the employment of filters, which is likely to be reflected in the user study (RQ7).


\subsection{Participants}
In the course of the user study, a group of 13 subjects was examined. 
The subjects included 12 males and one female. The majority of the subjects are currently enrolled in a Bachelor's or Master's program in Informatics. 
The majority of the subjects had prior experience with artificial intelligence. 
The subjects were aware of the objective of the study, which was to evaluate the performance of the AI system in comparison to conventional systems. 
The objective of the present study was to ascertain whether the artificial intelligence (AI) system is capable of producing superior results in the context of searching for products. 

\end{document}