\documentclass[../../submission.tex]{subfiles}
\begin{document}
\section{Discussion}
The objective of the paper is to propose the development of an artificial intelligence (AI) system designed 
to facilitate the search process for products by users.

\subsection{Use of Filter}
The user study revealed that the majority of users preferred to work with the filters first. 
This phenomenon may be attributed to the fact that contemporary systems are generally designed to assume that users will 
employ search filters during the search process. Consequently, users may find it unconventional to utilize solely the 
search bar to perform a search. The utilization of artificial intelligence by the majority of users does not align with 
conventional practices, as individuals have become accustomed to employing filters within online shopping platforms rather 
than utilizing the capabilities of AI directly. Following a series of inquiries, a conclusion was reached: The users have 
adapted to the system and are now able to discern whether the AI comprehends their intentions. Consequently, the majority 
of users have increasingly relied on the AI rather than the filters. Therefore, a period of adaptation is required for 
users to acclimatize to the KI. It is imperative to acknowledge the pivotal role of artificial intelligence in the 
integration of such systems within e-commerce platforms. The objective is to cultivate a conscious awareness among users.

\subsection{Improvement through AI}
The implementation of artificial intelligence has yielded notable enhancements in the realm of product search, 
thereby facilitating a more efficient and user-friendly experience for consumers. The artificial intelligence system demonstrated 
the capacity to automatically resolve spelling errors, thereby exhibiting enhanced tolerance for errors when compared to conventional 
systems. A further favorable aspect is that the AI was able to accurately interpret synonyms, thereby identifying the correct products.

\subsection{Limitations}
The implementation of AI could be exploited by users with relative ease, provided 
they have some degree of experience with AI. In the user study, for instance, a user 
effectively replaced the system with the search bar and issued commands through it. 

The search function implemented in the conventional system did not align with the prevailing 
technological standards. The search was exclusively based on textual descriptions, which is a limitation when attempting to emulate
existing web shops. It is therefore recommended that a search function be implemented that aligns with contemporary technological 
standards to facilitate the emulation of modern web shops.

\subsection{Future Work}
In light of the limited diversity observed in user studies, it would be advisable to undertake a study that involves fewer participants, 
particularly those with a background in information technology. Additionally, it would be beneficial to include older adult groups to 
assess the intuitiveness of the artificial intelligence interface. This approach would facilitate a comprehensive evaluation of the ease 
of use of the AI for different demographics.

A further investigation could be conducted in which the conventional system and the AI are compared on two separate pages. 
This would allow for the assessment of the efficiency and efficacy of AI utilization in a web shop. This approach could also address 
the filter bias identified in our user study.

Due to temporal constraints, a static template was employed, with its sole reliance on the description. In this instance, 
the integration of a state-of-the-art template could potentially enhance the comparison.

Another aspect that must also be mentioned in this context is protecting the system from being exploited by users. 
It is essential to ensure that users cannot exercise any control over the AI.

\end{document}