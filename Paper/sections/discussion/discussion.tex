\documentclass[../../submission.tex]{subfiles}
\begin{document}
\section{Discussion}
Our study set out to explore the potential of an AI-driven natural language interface (NLI) 
for enhancing e-commerce search experiences by dynamically generating SQL queries, as opposed 
to relying on traditional static templates and filtering mechanisms. The results, detailed in 
Section 4, provide compelling evidence that users find the AI-driven system more intuitive and 
satisfying (mean satisfaction score of 1.69, SD 0.48) compared to the conventional system (mean 3.62, SD 0.51). 
However, several limitations and biases in our methodology warrant a critical discussion to contextualize these 
findings and guide future improvements.

\subsection{User Satisfaction and Behavioral Shifts}
The high satisfaction with the AI-driven system, as reported in Section 4.1, aligns 
with our hypothesis from Section 1 that dynamically generated SQL queries could better 
capture user intent, reducing reliance on manual filters. Participants appreciated the AIs ability 
to handle synonyms (e.g., "Wiener Dog" interpreted as "Dachshund") and typographical errors, a capability 
absent in the static template approach. This finding echoes Section 2.2 discussion of modern NL-to-SQL systems 
leveraging large language models to interpret complex queries intuitively. However, this positive reception may be inflated by 
the participant pools composition—exclusively computer science students, with only one female participant 
among the 13. As noted in Section 3, this homogeneous group likely possesses above-average technical literacy, 
potentially skewing their comfort with an AI-driven interface over traditional filters. A more diverse sample, 
including non-technical users or older adults, might reveal different preferences, 
particularly given the adaptation period observed in Section 4.1, where users initially favored filters before shifting reliance to the AI.

\subsection{Role of Filters and Implementation Bias}
Section 4.2 revealed a nuanced picture: while 69.2\% of participants could manage without 
filters in the AI-driven system, a notable minority (23.1\%) preferred their retention for visual 
clarity (e.g., price sliders). This suggests that while the AI holds promise for reducing filter 
dependency—as hypothesized in Section 1.3—the transition may not be seamless for all users. A 
significant confounder here is the naive implementation of the static template, described in Section 
3.1 as a basic tokenization and OR-condition matching across breeds and descriptions. This rudimentary 
design, acknowledged in Section 4.1, likely underperformed compared to industry-standard static 
templates (e.g., those on Amazon or eBay), potentially exaggerating the AIs perceived superiority. 
A more sophisticated static baseline, incorporating weighted attributes or semantic matching, could 
have provided a fairer comparison. This limitation underscores a methodological 
bias that future iterations must address to validate the AIs advantages more robustly.

\subsection{Technical Feasibility and Prompt Design}
The technical feasibility of NL-to-SQL translation, a central research 
question from Section 1.3, was confirmed by the Claude API’s ability to 
generate accurate SQL queries from diverse user inputs, as detailed in 
Section 3.1. However, a significant security vulnerability emerged in 
Section 4.3: the AI translated malicious inputs, such as „delete all tables,“ 
into executable SQL queries. Unlike the static template, which avoids 
such risks through its predefined structure, this flaw underscores a 
critical gap in our implementation and echoes the SQL Injection concerns 
raised in Section 1.1. The system prompt for the Claude API, outlined in 
Section 3.1, was rudimentary—limited to the database schema and a base 
query—lacking explicit constraints 
(e.g., restricting operations to SELECT statements). This 
oversight clashes with Section 2.2’s emphasis on transparency
and safety as prerequisites for modern NL-to-SQL systems. A more 
robust solution would have integrated our proposed blacklist 
configuration (Section 4.3), filtering 
out hazardous commands like DELETE or DROP. Additionally, instructing 
users to test the system with adversarial inputs akin to „delete all tables“ 
could have exposed further weaknesses, but time constraints prevented 
this enhancement. These refinements would have strengthened both 
security and practical applicability, aligning the system more 
closely with real-world requirements.

\subsection{Correction Strategies and Missing Metrics}
Section 4.4 outlined user-suggested strategies like dynamic filter adjustment and query confidence 
indicators, addressing the third research question from Section 1.3 about refining inaccurate results. 
These suggestions align with Section 2.3 emphasis on usability and feedback in NL-to-SQL systems, 
enhancing transparency and trust. However, our study lacked quantitative metrics to assess filter 
interaction rigorously, such as a counter, tracking filter usage frequency or duration. This omission, stemming from the study design in Section 3, limits our ability to 
statistically correlate filter reliance with satisfaction or adaptation, weakening claims about 
behavioral shifts (Section 4.1). Future work should integrate such metrics to provide a data-driven basis 
for evaluating the AIs potential to supplant filters, as proposed in Section 5.4.

\subsection{Additional Limitations}
Beyond the aforementioned biases and limitations, additional weaknesses emerged. 
The small sample size (n=13), acknowledged in Section 4.1, precludes statistical 
significance testing, rendering our findings exploratory rather than conclusive. 
The choice of dogs as a product category (Section 3), while practical, may not generalize to 
more complex domains like electronics, where interdependent attributes (Section 1.2) could challenge the AIs 
query generation accuracy. Furthermore, the side-by-side interface design (Section 3.1) may have introduced a contrast effect, where the AIs results appeared more favorable simply due to the static systems visible shortcomings, 
a potential bias not addressed in our analysis.


\end{document}