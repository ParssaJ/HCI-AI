\documentclass[../../submission.tex]{subfiles}
\begin{document}
    \begin{abstract}
        Online shopping platforms often struggle to meet diverse user search 
        needs due to the rigidity of traditional SQL templates and filtering 
        mechanisms, leading to suboptimal user experiences. This paper 
        explores an AI-driven natural language interface (NLI) that 
        dynamically generates SQL queries from user inputs, aiming to 
        enhance search intuitiveness and reduce dependency on manual 
        filters. We developed a prototype e-commerce interface focused 
        on a dog webshop, comparing AI-generated queries 
        (via the Claude API) against a static template approach 
        in a user study with 13 participants. Results show high 
        user satisfaction with the AI system (mean 1.69, SD 0.48) 
        compared to the conventional system (mean 3.62, SD 0.51), 
        driven by its ability to interpret intent and handle synonyms 
        (e.g., "Wiener Dog" as "Dachshund"). While 69.2\% of users could 
        forgo filters, a minority favored their retention for visual 
        clarity, highlighting a nuanced role for traditional mechanisms. 
        Technical feasibility was demonstrated, though security 
        vulnerabilities (e.g., unintended query execution) necessitate 
        safeguards like blacklists. User-suggested 
        correction strategies, such as dynamic filter adjustments and 
        confidence indicators, to further enhance transparency and trust. 
        Despite limitations—including a small, tech-savvy sample and a 
        basic static baseline—our findings suggest NL-to-SQL systems 
        can transform e-commerce search, with broader implications for 
        human-computer interaction, provided usability and security 
        challenges are addressed.
    \end{abstract}
\end{document}
