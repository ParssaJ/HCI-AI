\documentclass[../../submission.tex]{subfiles}
\begin{document}
\section{Introduction}
In order for a user to interact with a website, an interactive element such as 
an input text is typically provided. Many of these input text-elements work by
having a predefined SQL-Query-Template, for which the users input gets concatenated.
This query then gets passed to the Database-Management-System (DBMS), which executes
it and returns the result. Although this approach seems simple at first, it has 
some pitfalls which need to be addressed, as it is prone to attacks like SQL-Injections.
According to the Open Worldwide Application Security Project (OWASP), which is a 
non-profit organization that aims to improve the security of software projects,
SQL-Injections are still in the Top 10 Security Risks of 2021 and it is likely that 
they will remain in the Top 10 of 2025, taking into account that they were the number
one attack in the Top 10 of 2017 \cite{owaspfoundationOWASPTopTen2024}.
\subsection{Generating SQL-Queries dynamically}
Instead of relying on predefined SQL-Templates, we would like to propose the 
idea of generating SQL-Queries dynamically using Machine-Learning-Techniques.\\
By generating SQL-Queries dynamically, the users wishes provided in the input text can 
be taken into account by the AI automatically, providing more specific search results initially.
Given that the AI performs well enough, this could lead to future websites dropping 
filtering mechanisms entirely, such as filtering by specific brands or other features,
as the AI would reflect the users wishes automatically in the generated Query, 
further enhancing the Human-Computer Interaction. This new kind of interaction is highlighted
in Fig. 1.
\subsection{Why a privacy-preserving AI is necessary}
Although the idea of generating SQL-Queries automatically sounds intriguing, there 
are some practical challenges that have to be taken into account, when deploying the AI 
into a software project. Under the assumption
that the AI performs well and that it is unrestrained, nothing would prevent the user
from requesting search results from the AI, that contain confidential information like 
passwords, email-addresses or other criteria, which poses a similar threat that 
are resembled by SQL-Injections. Arguably, this would make the threat of SQL like 
Injections even worse, as users with no technical background are now able to query 
for confidential data as well. To mitigate that issue, we enforce a privacy-preserving AI,
with the goal to never output SQL-queries that could disclose confidential data.
\subsection{Gathering Data}
TODO
\end{document}