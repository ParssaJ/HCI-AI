\documentclass[../../submission.tex]{subfiles}
\usepackage{graphicx,wrapfig,lipsum}

\begin{document}
\section{Introduction}
Online shopping platforms like Amazon and eBay have become the go-to sources for a vast range of products. 
However, users often face difficulties when searching for specific items, especially products with more complex or technical specifications, 
such as electronic devices, computer components, or specialized tools. These challenges can lead to frustration and even cause users to switch platforms in 
search of a better experience. This frustation stems from individual differences in search behaviour, which varies widely among users.
Some rely on generic search terms, such as "laptop," while others prefer highly specific queries, like "Dell XPS 15 9530 with Intel Core i7 and 32GB RAM." 
In a preliminary analysis conducted as part of this research, we observed that search behavior splits roughly evenly between these two approaches, a 
finding derived from user interactions with our prototype interface described later in this paper, making it essential for e-commerce platforms to optimize their search functionalities to 
accommodate diverse user needs. At the same time, this presents a challenge, as creating a platform that satisfies all users equally is difficult.\\
While our investigation focuses on e-commerce due to its widespread use and direct impact on user satisfaction and business success, the proposed approach could also 
benefit other domains, such as medical databases or academic research platforms, where complex queries and diverse user needs are similarly prevalent.\\
In addition to search queries, filtering options play a crucial role in the online shopping experience. Filters help users refine their results based on 
criteria such as price, size, brand, technical specifications, or customer ratings. However, ineffective or overly complex filtering systems can hinder rather 
than enhance the search process. If filters fail to match user expectations, prove difficult to navigate, or do not efficiently narrow down results, users may
abandon their search altogether. Therefore, improving search functionalities is essential to enhance user experience, increase customer satisfaction
and ensure platform loyalty.\\
Building upon these challenges, we propose an alternative approach to traditional search mechanisms. Instead of relying on predefined SQL templates, we suggest 
dynamically generating SQL queries using machine learning techniques. By leveraging AI to generate queries in real time, the system can automatically interpret 
the user’s intent based on their input, providing more precise and relevant search results initially.
If the AI achieves high accuary, this approach could reduce or even eliminate the need for manual filtering by embedding user preferences directly into the generated queries. 
This shift could significantly enhance human-computer interaction, making the search process more intuitive, efficient, and user-centric.
To determine whether such a scenario is truly possible and to understand how users would interact with this approach, we conducted a study. 
For this purpose, we developed a prototype interface that accepts natural language search queries, allowing us to assess user behaviour and 
compare the effectiveness of dynamic query generation against traditional static templates.

\subsection{The conventional search mechanism}
The Open Worldwide Application Security Project (OWASP) regularly updates its top 10 list of web application security risks.
SQL Injection attacks have consistently remained in this list across multiple iterations, including the 2017 and 2021 versions,
largely due to the widespread use of custom SQL query templates in web applications\cite{owaspfoundationOWASPTopTen2024}. These templates
follow predefined schemas where user input, such as search queries, is inserted into predetermined placeholders. This approach
presents multiple significant limitations. These templates are vulnerable to SQL Injection attacks (SQLIA) if 
not properly secured. For example, a poorly sanitized input like "laptop; DROP TABLE products;" could compromise the entire database. 
Additionally, their search capabilities are inherently restricted, as the searchable attributes 
or columns must be explicitly defined by developers based on specific requirements. While these templates excel at 
simple keyword matching, they 
struggle with natural language queries due to their rigid, predefined structure. Although this keyword-based 
approach can efficiently process basic search terms, users typically need to rely on additional filtering 
mechanisms to narrow down their results to find exactly what they are looking for. However, these filtering 
systems come with their own set of limitations.

\subsection{Limitations of traditional filtering mechanisms}
Traditional e-commerce platforms typically implement filtering systems through a combination of 
predefined dropdown menus, checkboxes, and range selectors. While these mechanisms provide basic 
functionality, they present several significant challenges. First, the static nature of these 
filters requires regular maintenance and updates to accommodate new product categories and attributes. 
Second, the presentation of filter options often follows a one-size-fits-all approach, disregarding 
individual user preferences and shopping patterns. For instance, technical users searching for computer 
components might prefer detailed specification filters, while casual shoppers might find such granularity 
overwhelming. Furthermore, traditional filtering systems often struggle with semantic relationships between 
different product attributes. A user searching for a "gaming laptop" might need to individually select filters
for high RAM (e.g., 16GB) and a fast process (e.g., Intel Core i7), despite these being inherently tied to 
gaming performance. This disconnect between user intent and filtering 
capabilities becomes particularly problematic when dealing with complex product categories that 
have numerous interdependent specifications, such as electronic devices or specialized equipment.

\subsection{Moving beyond traditional search limitations}
These limitations of conventional search templates and filtering mechanisms highlight 
the need for more sophisticated approaches that can adapt to diverse user needs. 
Modern developments in database querying, machine learning, and natural language processing 
offer promising directions for adressing these challenges through more flexible and 
intelligent search solutions.\\
Based on the identified challenges in e-commerce search mechanisms, we formulate three main research 
questions that guide our investigation.
\begin{enumerate}
    \item We examine how natural language interfaces affect user search behavior and satisfaction 
    compared to traditional keyword-based approaches.
    \item We investigate the technical feasibility of translating diverse natural language queries into SQL statements.
    \item We analyze user preferences and requirements regarding correction mechanisms in AI-driven search systems. 
    This investigation focuses on understanding which types of refinement options users would find most helpful when 
    initial search results don't meet their expectations, and how these correction possibilities influence their trust 
    in the system.
\end{enumerate}

\end{document}